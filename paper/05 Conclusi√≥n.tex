\section{Conclusión}

A lo largo de este documento, se describió la problemática que presentan las ecuaciones diferenciales que no pueden ser resueltas de forma analítica, resaltando la necesidad de métodos numéricos precisos y confiables para la aproximación de esta solución.

El documento se enfocó en los métodos de Runge-Kutta, más específicamente en el de grado 4, describiendo su funcionamiento y presentando una visualización de él. Se culminó con una aplicación concreta en la que puede ser aplicado este método numérico: la aproximación a la solución de la ecuación de Van der Pol, que no puede ser encontrada analíticamente.

Para propósitos de la aplicación con Van der Pol, se desarrolló un programa en Python que aplica el método para resolver la ecuación. Al utilizarlo para resolver tres casos, se comprobó que el programa es efectivo para encontrar la solución, pudiendo también observar la naturaleza numérica de esta solución.

Todo esto fue de gran ayuda para entender cómo funciona un método numérico aplicado a las ecuaciones diferenciales, así como la relativa sencillez de estos métodos y especialmente simple implementación en lenguajes de programación modernos. También se resaltó la importancia de estos métodos numéricos, pues se utilizó el método destacado, RK4, para encontrar las soluciones un tipo de oscilamiento que se presenta en muchas disciplinas modernas.
