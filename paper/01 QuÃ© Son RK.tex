\section{¿Qué son los Métodos de Runge-Kutta?}

Los métodos de Runge-Kutta son una familia de métodos numéricos iterativos desarrollados por los matemáticos alemanes Carl Runge y Wilhelm Kutta alrededor de los inicios del siglo XX.\@

Estos métodos tienen como objetivo obtener una curva aproximada a la solución de un problema de valor inicial que involucra una ecuación diferencial \textit{ordinaria}, especialmente cuando se trata del tipo de problemas cuya solución analítica es difícil o imposible de obtener, como ya fue establecido en la introducción de este texto. Debido a su relativamente rápida convergencia en relación a otros métodos, los métodos de Runge-Kutta son ampliamente preferidos, a pesar de que son computacionalmente más demandantes,

De forma similar a otros métodos numéricos utilizados para las ecuaciones diferenciales, los métodos de Runge-Kutta son iterativos, lo que quiere decir que una primera estimación es utilizada para obtener una segunda más precisa y así sucesivamente.
